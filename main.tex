%-----------------------------------------------------------------------
%
%   UFRJ  - Universidade Federal do Rio de Janeiro
%   COPPE - Coordenação dos Programas de Pós-graduação em Engenharia
%   PEE   - Programa de Engenharia Elétrica
%
%
%   Projeto EMMA - 
%
%                                                        10/jul/15, Rio
%                                                        Ramon R. Costa
%----------------------------------------------------------------------
%
%  Compilar usando PDFLaTeX
%
%----------------------------------------------------------------------
\documentclass[12pt,a4paper]{article}
\usepackage{macros/ROSApackages} 

\input{macros/ROSAsettings}
\input{macros/ROSAmacros}

%\def\PATH{file:c:/Users/Ramon/My Documents/projetos/2015/Projeto EMMA}

\begin{document}
%---------------------------------------------------------------------
\include{00_Frontpage}

%---------------------------------------------------------------------
%\tableofcontents

\newpage%
%---------------------------------------------------------------------
\section{Identificação}

%-----------------------------------------------------------------------
%
%   UFRJ  - Universidade Federal do Rio de Janeiro
%   COPPE - Coordenação dos Programas de Pós-graduação em Engenharia
%   PEE   - Programa de Engenharia Elétrica
%
%
%   Projeto EMMA - Metodologia para revestimento robótico de turbinas in situ
%
%   Identificação
%                                                         Ramon R. Costa
%                                                         07/jul/15, Rio
%-----------------------------------------------------------------------
%\section{Identificação}

\dado{Título}{
  EMMA - Metodologia para revestimento robótico de turbinas \textit{in situ} \\
}

\dado{Proponente}{
  Universidade Federal do Rio de Janeiro (UFRJ) \\[2mm]
  Fundação Coordenação de Projetos, Pesquisas e Estudos Tecnológicos (COPPETEC) \\
}

\dado{Contratante}{
  ESBR - Energia Sustentável do Brasil S.A. \\
}

\dado{Execução}{
  Grupo de Simulação e Controle em Automação e Robótica (GSCAR) \\
}

% \dado{Contrato}{
%   0000000000 \\
% }

% \dado{P\&D ANEEL}{
%   6631-0002/2013 \\
% }

\dado{COPPETEC}{
  N.D. \\
}

\dado{Início}{
  N.D. \\
}

\dado{Prazo}{
  N.D. \\
}

\dado{Orçamento}{
  N.D. \\
}

\dado{Coordenador}{
  N.D. \\
}

\dado{Gerente}{
  N.D. \\
}

%---------------------------------------------------------------------
\fim

\newpage%
%---------------------------------------------------------------------
\section{Sobre a visita}
Foram dois dias de visita à Rijeza, 08 e 09 de Julho de 2015:

Dia 08/07/2015:
\begin{itemize}
  \item Apresentação realizada por Ramon/Renan: requisitos do processo HVOF,
  restrições do ambiente e soluções conceituais para os três acessos da turbina de
  Jirau. A solução conceitual pelo acesso superior foi a mais elaborada até
  o momento.
  \item Apresentação realizada pela Rijeza: requisitos do processo HVOF,
  cavitação e erosão das pás de Jirau, soluções para exaustão do processo e
  instalação dos componentes.
\end{itemize}

Dia 09/07/2015:
\begin{itemize}
  \item Visita às instalações e aplicação de hardcoating.
  \item Reunião para resumo, feedback e próximos passos.
\end{itemize}

%---------------------------------------------------------------------
\section{Informações}

\subsection{Sobre Jirau}

\begin{itemize}
  \item Todas as turbinas de Jirau têm uma escotilha superior semelhante. Éder
  fará a verificação do diâmetro da escotilha superior na turbina chinesa e providenciará os desenhos.

  \item A estrutura do andaime pode ser soldada dentro do aro-câmera. Esse é um procedimento usual.
  
  \item O distribuidor pode ser aberto durante a manutenção da turbina. Isso é
  importante para a solução de exaustão.
  
  \item Para evitar o problema ambiental causado pelos peixes que invadem o
  túnel durante as paradas, a comporta vagão é fechada juntamente com o
  distribuidor. Isso cria uma correnteza durante o fechamento que impede a
  entrada dos peixes. Com essa nova informação, a opção de acesso pela
  jusante deve ser abandonada, devido à correnteza durante o procedimento de
  fechamento da comporta vagão.
  
  \item São 30 dias para desmontar a turbina e 30 dias para a montagem.
  
  \item A pá só pode ser girada antes do desligamento da máquina. Após uma
  angulação ser escolhida, deveremos trabalhar com ela até o final.

\end{itemize}

\subsection{Sobre o processo de revestimento}

\begin{itemize}  
  \item Para revestir uma turbina completa são necessários cerca de 500\,Kg de
  pó metálico. O rendimento do processo de revestimento é de 50\%. Isto é, 250\,Kg de pó metálico serão perdidos.
  
  \item A potência gerada pelo processo é de cerca de 1.5\,MW ($\approx$ 2.000\,HP).
   
  \item A ventilação necessária para a refrigeração é bem grande.
  
  \item As pistolas que existem no mercado possuem praticamente o mesmo peso.
   
  \item A RIJEZA planejava utilizar a escotilha superior para a exaustão de ar.
Éder sugeriu abrir o distribuidor para a ventilação. Existe uma outra escotilha
do outro lado.

  \item A RIJEZA avaliou a perda de carga se o equipamento for instalado abaixo
  da turbina devido ao grande comprimento das mangueiras e cabos, e a altura.

  \item A RIJEZA achou boa a opção de acesso pela escotilha superior. Isso resolve o problema da perda de carga.

  \item Esclarecida a dúvida sobre a precisão necessária/possível. O mecanismo
  não será tão preciso quanto o manipulador. A diferença de uma trajetória para
  a outra é de 3\,mm.

  \item Sugestão: Checar o software SprutCAM para a simulação de cinemática.

  \item Sugestão: Checar o sistema laser vision ARC-EYE. Pode ser uma opção para
  a calibração e identificação de bordas.

  \item Ideia alternativa para o shutter: utilizar um switch para desviar o pó.
  
  \item Ideia: projetar um container com todo o equipamento. O container deve se
  apoiar sobre o aro câmera sem interferir no acesso já instalado.

  \item Ambiente: os resíduos podem ser abandonados? Questão a ser investigada.

  \item Ideia: a turbina poderia ser lavada ao final do procedimento de
  revestimento e haver algum tipo de separação entre a água e o pó, que
  decanta.

  \item Segurança: raio mínimo de curvatura das mangueiras deve ser observado.
  Cuidado para não quebrar o conector.

  \item Segurança: será necessário projetar um mecanismo para acender a tocha
  remotamente.

  \item A RIJEZA utiliza robôs da ABB. A assitência técnica é rápida. Considerar
  a utilização de robôs da ABB no caso de acesso pela escotilha inferior.

  \item Existe um processo de revestimento que usa querosene. É mais complexo e
  consome mais energia para refrigeração. Possivelmente, não deve valer a pena.

\end{itemize}

\subsection{Algumas conclusões}

\begin{itemize}
  \item Abandonar a opção de acesso pela jusante.

  \item Solução convergindo para usar um manipulador de pequeno porte e um procedimento de revestimento por fases.

  \item Tendência: focar na opção de acesso pela escotilha superior. Verificar
  acesso dos cabos com passagem por dentro da base móvel da escotilha superior.
  
  \item Abrir o distribuidor para auxiliar a ventilação.

  \item A logística para a movimentação é um ponto chave. É um ponto favoravel
  importante a ser considerado na avaliação do acesso pela escotilha superior.

\end{itemize}


\subsection{Próximos passos}

\begin{itemize}
  \item Avancar com o conceito do acesso inferior. Redirecionar o Gabriel.

  \item Estudar com mais detalhes o uso do robô pequeno com um processo de revestimento em fases.

  \item Estudar o problema de aceleração e desaceleração do robô.
  
  \item Estudar o problema do sincronismo.

  \item Estudar o problema da calibração.

  \item Falar com o Toni sobre o problema de calibração e simulação dinâmica.

\end{itemize}




\vspace{20mm}%
\parbox[t]{70mm}{
  \centering
  \includegraphics[width=65mm]{assinatura/assinatura-digital.jpg} \\[-4mm]
  \rule[2mm]{70mm}{0.1mm} \\
  \ramon \\[1mm]
  Coordenador do Projeto \\
}



%---------------------------------------------------------------------
\end{document}
